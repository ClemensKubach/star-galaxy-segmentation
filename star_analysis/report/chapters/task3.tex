We decided to try classical AI and deep learning for our models.

\section{Classical AI}
For classical AI, we used an SVM-Classifier and a Logistic-Regression. As input we used individual pixels. Both approaches were evaluated on the same data.

\subsection{SVM-Classifier}
Due to computational limits on our side, we just used 10 images to train on. To further reduce the amount of data, we only considered pixels where one channel $>127$. This threshold is a hyperparameter, which we did not tune further because of bad initial performance.

The code can be found in \path{star_analysis/model/svm.py}

\subsection{Logistic Regression}
This approach is fast enough to learn in batches of 20 images. Also, we do not need to down-sample the data to make it feasible to train. Therefore, this approach learns from much more data. We used 100 images in total for training.

For code, see \path{star_analysis/model/logistic_regression.py}

\subsection{Performance}
\begin{tabularx}{\linewidth}{mmmm}
    \hline
    Model               & Accuracy & Macro-Precision & Macro-Recall \\
    \hline
    SVM-Classifier      & $0.99$   & $0.33$          & $0.33$       \\
    \hline
    Logistic Regression & $0.99$   & $0.33$          & $0.33$       \\
\end{tabularx}

\subsection{Conclusion}
The way we applied the classical methods did not result in a classifier which predicts the classes properly. The models mostly predict everything to be sky or background.
